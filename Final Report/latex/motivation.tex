\section*{Motivation}

%We have chosen to do a project related to the current COVID-19 pandemic. 

We are interested in the analysis of political discourse on Twitter in the US with respect to the ongoing \texttt{COVID-19} pandemic. The motivation behind this analysis is to identify political narratives that harnessed more support among the users of the micro-blogging site and that were pushed by different news agencies owing to their implicit political bias. To build the training corpus, we will utilize tweets pertaining to \texttt{COVID-19} made by the US Governors hailing from Democratic or Republican parties, allowing us an efficient way to identify and label tweets representing discourse pushed from either side of the political spectrum to deal with the same crisis. Using the trained model, we intend to analyze the Twitter accounts of all major US news outlets to identify the support for either of these political discourses and whether their stance evolved over time. For possible future work, another application of the model could be used to track real-time twitter sentiments in response to real-world events such as, how do the postings on the platform evolve following events, such as the White House briefings.


 %We may also apply our model to news articles to see if different news channels or sites have an implicit political bias.
 
%We will be analyzing tweets pertaining to coronavirus made by U.S. Governors to see if we can predict their political party. The main motivation behind this idea is to see if there is a difference in how members of different political parties respond to the same crisis. We may also apply our model to news articles to see if different news channels or sites have an implicit political bias.

%Basically, the model will classify randomly acquired tweets from users in the US and classify them based on political sentiments present in the text. The motivation/application of such a system could be to 1) perform longitudinal analysis on history of tweets since day 1 of covid19 cases and discover which political sentiments have dominated discussions 2) to develop an application that could track real-time twitter sentiments in response to real-world events such as, how does the postings on the platform evolved following events such as CDC/Trump briefings and so on.
